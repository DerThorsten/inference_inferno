\documentclass{article}

\usepackage{bbm}
\usepackage{amsmath}
\setlength\parindent{0pt}

\newcommand{\norm}[1]{\left\lVert#1\right\rVert}

\begin{document}


\section{Pixel Prediction}

Use either auto-context or CNN predict
semantic classes and edge probability map


\section{Boundary/Face Prediction}

Use pixel prediction results and a
handful of filter results to accumulate
features for each face.

The filters will / should be simple blockwise callable filters as:
\begin{itemize}
\item hessian of gaussian eigenvalues
\item structure tensor eigenvalues
\item$ \ldots$
\end{itemize}

Use a default random forest to predict boundary
probability map.






\section{Features for Structured Learning}
First step is a simple (quantile based)
normalization of the raw features 
which have been used for the unstructured learning.
The result should be features in -1,1 or 0,1.

The random forest face probabilities should also be used
as a feature. No normalization is needed here.

To make the features more powerful, we try
to emulate the kernel trick by approximating an
rbf kernel or a polynomial kernel
\begin{itemize}
\item Nystroem Approximation of kernel
\end{itemize}


\section{Structured Learning}

Let $G$ be a graph $G = (V,E)$.
For each variable  we have a node label $y_i \in |V|$.
Therefore, each variable has as many labels as there are variables
such that we can encode any partitioning.
The multicut objective as a graph labeling problem is given by:
\begin{align*}
    y_{map} &=   
        \textbf{argmin}_{y} 
                \sum_{ uv \in E} 
                    \beta_{uv} \cdot \big[ y_u \neq y_v \big]
\end{align*}

A negative $\beta_{uv}$ is a prior/desire that node $u$ and $v$ should be
in different clusters, a positive $\beta_{uv}$ is a prior/desire to
put $u$ and $v$ in the same cluster.

To learn this objective we replace with $\beta_{uv}$
with $w^Tx_{uv}$ where $w$ is the structured learning weight 
vector, and $x_{uv}$ the feature vector for the edge $uv$.
With  $\beta_{uv} = w^Tx_{uv}$ we get:
\begin{align*}
    y_{map} &=   
        \textbf{argmin}_{y} 
                \sum_{ uv \in E} 
                    w^Tx_{uv} \cdot \big[ y_u \neq y_v \big] \\
    y_{map} & =   
        \textbf{argmin}_{y} 
                \sum_{ uv \in E} 
                    w^Tx_{uv} \cdot \big[ y_u \neq y_v \big]
\end{align*}
Let us define $\textbf{Y}(w)$ as the MAP labeling for a given weight:
\begin{align*}
    \textbf{Y}(w) & =   
        \textbf{argmin}_{y} 
                \sum_{ uv \in E} 
                    w^Tx_{uv} \cdot \big[ y_u \neq y_v \big]
\end{align*}



Let $DS$ be the dataset.
We will use $i$ to index a certain training example.
\begin{align*}
    w_{opt} = 
        argmin_{w} 
        \hspace{0.2cm} \norm{w}^2 + 
        C \cdot 
        \sum_{i \in |DS|}  
        \Delta(y^i_{gt},\textbf{Y}^i(w))
\end{align*}

\subsection{Weight Initialization}


\end{document}
